\begin{frontmatter}

    \title{Graph Attention Network (GAT) based Branching in
Combinatorial Optimization Problems}

    \author[label1]{Gökhan Murali\corref{mycorrespondingauthor}}
    \ead{gmurali21@ku.edu.tr}
    \author[label1]{Prof. Metin Türkay}

    \affiliation[label1]{organization={Koc University},
             addressline={Rumelifeneri Yolu, Sariyer},
             city={Istanbul},
             postcode={34450},
             %state={NSW},
             country={Turkiye}}

    \cortext[mycorrespondingauthor]{Corresponding author}

    \begin{abstract}


        In this study, the main aim is to imitate the Strong Branching strategy during the branching phase, which is one of the most critical components of the Branch \& Bound algorithm used for solving Combinatorial Optimization problems, by implementing a Graph Attention Network (GAT)-based method.
        Strong Branching is an effective strategy in terms of the number of nodes, keeping the search tree short.
        However, it is highly time consuming because it solves the linear programming problem twice for each branching candidate variable at each node.
        To eliminate the time cost of the Strong Branching strategy, this study attempts to learn a function implementing the GAT technique that can make Strong Branching-like decisions in a shorter time.

        In the literature, there are studies that successfully imitate the Strong Branching strategy using Graph Convolutional Neural Network (GCNN). In the GCNN method, all neighbouring nodes have the same importance for a node.
        In contrast, the GAT  architecture allows neighbouring nodes to have different levels of importance for a node.
        Therefore, it is hypothesized that GAT-based methods will yield better results.

        Experiments conducted in this study have shown that the GAT architecture provides decisions closer to the Strong Branching strategy compared to the GCNN architecture.
        GAT-based methods enable problems to be solved with fewer nodes compared to GCNN.
        In summary, GAT is a promising tool for imitating the effective yet slow Strong Branching strategy.
        Supplementary code for this study can be found at \url{https://github.com/GokhanMurali/learn2branchbyGAT}.
    \end{abstract}


    \begin{keyword}
        Mixed Integer Programming \sep Branch \& Bound \sep Strong Branching \sep Machine Learning \sep Neural Networks \sep Graph Neural Network (GNN) \sep Convolution \sep Message Passing \sep Graph Attention Network (GAT) \sep Graph Convolutional Neural Network (GCNN)
    \end{keyword}

\end{frontmatter}
